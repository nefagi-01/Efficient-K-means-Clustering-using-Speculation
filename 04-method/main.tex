This section describes the algorithm from a higher level, and from the accuracy/convergence perspective. 

Basically, mention very briefly, referring to the related work how do the following behave:
\begin{itemize}
    \item naive k-means
    \item better initialization
    \item approximate k-means
\end{itemize}

Then, you could say that we take the best of all the worlds using speculation, and do this adaptively at runtime.

That is, instead of making an apriori initialization, you do this through sampling to converge faster. Yet, due to repair you obtain a correct, precise answer. This also avoids initialization time, and just processes samples rather than full data (could be costly, as the google paper with sampling proposed). 

Then, you present the solution step by step, how you start from the naive kmeans, what are steps A and B, why do they have a dependency, and why resolving this dependency helps the convergence (in the next section there will be time for performance discussion, here at an abstract level describe the approach).

Then, add a few graphs to indicate how long steps A and B usually take in the first, naive speculative case.

Then, you can discuss pitfalls and how and why you improve sketching. Here you can use those convergence graphs, but just as a motivation, there will be an evaluation section for in-depth discussion.