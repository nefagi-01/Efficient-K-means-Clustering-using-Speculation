The previous section provided an overview of the K-Means clustering algorithm. We discussed the advantages and disadvantages of various approaches and how they can be applied in different scenarios. We now propose a new approach based on Speculation which aims to solve the trade-offs observed in the previous methods.
We define now clearly the objectives we are trying to reach with K-Means Clustering Using Speculation, and how we are going to measure it.
Firstly, our goals are:
\begin{enumerate}
    \item to reduce the number of steps needed to converge,
    \item to reduce the overall time needed to converge,
    \item to achieve a comparable quality of final results to the case in which we work only with the entire dataset,
    \item and to have the capacity to converge to a global minimum without a particular initialization.
\end{enumerate}
To show these results, we will observe and compare the evolution of the clusters' inertia during the Basic K-Means algorithm and the one using Speculation. Additionally, we will compute an estimate of the optimal clustering and its respective inertia.
This will be essential to understanding if our method is reducing the inertia faster than existing methods, and to verify where our method is converging with respect to the global minimum and, in particular, if it is able to reach it.