In this study, we tried to improve performances of K-Means without sacrificing the accuracy of the final results. To address this, we used the Speculation technique from \cite{Sioulas:282304}, which allowed us to break the cyclic dependency in K-Means, by creating two threads of execution, one for speculating future centroids, and the other for ensuring accurate convergence. Our findings showed that it is indeed possible to significantly reduce the number of stages (see Section \ref{section:breaking_dependencies} 
 and \ref{section:evaluation_reduced_stages}) required to converge without sacrificing the quality of the results. Additionally, we demonstrated how the centroid resampling technique can be used to escape local minima (see Section \ref{section:escaping_local_minima} and \ref{section:evaluation_escaping_local_minima}) without incurring initial time costs due to specific initialization methods. Finally, we proposed different approaches for scaling up the Speculation technique, which could be further explored in future work.